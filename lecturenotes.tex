\documentclass[11pt,a4paper]{report}
\fontfamily{cmss}

\begin{titlepage}
\title{BU5526 - Portfolio Analysis \\ Lecture Notes}
\author{Rodrigo Miguel}
\date{\today}
\end{titlepage}



\begin{document}
\maketitle
\tableofcontents

\chapter{Lecture 1}
\section{Asset Pricing}
\paragraph{Asset}
Something valuable that an entity owns, benefits from, or has use of, in generating income.
\paragraph{Asset Pricing}
Needed to buy/sell at a fair price.
\begin{itemize}
    \item Obtained by discovery process:
        \begin{itemize}
            \item Demand and supply forces.
        \end{itemize}
    \item The price of an asset is simply the current value of its cash-flows.
    \[ PV = \frac{FV_1}{(1+r)^1} + \frac{FV_2}{(1+r)^2}+ ... + \frac{FV_n}{(1+r)^n} \]
        \item What we need:
        \begin{itemize}
            \item Prediction of cash flows;
            \item Discount rate.
        \end{itemize}
\end{itemize}

\paragraph{How to price a stock:}
\begin{itemize}
    \item Dividend Discount Model;
    \item FCF Model;
    \item Multipliers and Comparables approach.
\end{itemize}
Being able to estimate the required \textbf{rate of return} is everything you need to calculate the \textbf{value of any asset} - once you have predicted \textbf{future cash-flows}, including:
\begin{itemize}
    \item Estimating the value of a stock;
    \item Estimating the value of a firm.
\end{itemize}

\section{Portfolio Management}
\textbf{Prudent} administration of \textbf{investable} (liquid) assets, aimed at achieving an \textbf{optimum risk-reward ratio}.
\section{Mathematical Concepts}
\paragraph{Mean} an average of different observations.
\begin{itemize}
    \item Useful to describe a population;
    \item \textbf{Arithmetical}: sum of observations divided by the number of observations (if with equal weights).
    \[\overline{X} = \frac{X_1 + X_2 + ... + X_n}{n}\]
    or
    \[\overline{X} = W_1 \times X_1 + W_2 \times X_2 + ... + W_n \times X_n\]
    \item \textbf{Geometrical}: product of observations divided by the number of observations.
    \[\overline{X}_G = \sqrt[n]{X_1 \times X_2 \times ... \times X_n}\]
    or
    \[\overline{X}_G = \sqrt[n]{X_1^{W_1} \times X_2^{W_2} \times ... \times X_n^{W_n}} \]
\end{itemize}
There is a reason to use arithmetical or geometrical mean.
\begin{itemize}
    \item \textbf{Equally weighted}: when all the observations have the same importance.
    \item \textbf{Unequally weighted}: different importance for different observations.
\end{itemize}

\paragraph{Variance} seen as an extension of the mean.
\begin{itemize}
    \item Dispersion to the mean (higher/lower):
    \begin{itemize}
        \item Average of differences from the mean.
        \[\sigma_x^2 = \frac{(X_1 - \mu_X)^2 + (X_2 - \mu_X)^2 + ... + (X_n - \mu_X)^2}{n}\]
    \end{itemize}
    
\end{itemize}

\paragraph{Important Statistical Concept}
\begin{itemize}
    \item The formula is correct if we possess all the data on the population;
    \item If we only have a sample, we need to reflect "impreciseness" by removing "one degree of freedom".
    \[\sigma_x^2 = \frac{(X_1 - \mu_X)^2 + (X_2 - \mu_X)^2 + ... + (X_n - \mu_X)^2}{n-1}\]
\end{itemize}
This will \textbf{always} be the case in finance.

\paragraph{Standard Deviation}
\[\sigma_X = \sqrt[2]{\frac{(X_1 - \mu_n)^2 + (X_2 - \mu_n)^2 + ... + (X_n - \mu_n)^2}{(n-1)}} \]

\paragraph{Skewness}
Brings back the sign. A positive skewness means more positive-value and reversely.
\[\sigma_X^3 = \frac{(X_1 - \mu_n)^3 + (X_2 - \mu_n)^3 + ... + (X_n - \mu_n)^3}{n} \]

\paragraph*{Kurtosis}
Outweighs extremes - dropping the sign.
\[\sigma_X^4 = \frac{(X_1 - \mu_n)^4 + (X_2 - \mu_n)^4 + ... + (X_n - \mu_n)^4}{n} \]
A large Kurtosis means a lot of extreme values.
\begin{itemize}
    \item To get a \textbf{Meaningful estimate}: needs to provide \textbf{excess} kurtosis.
    \item Kurtosis of a normal distribution is 3.
\end{itemize}

\paragraph{Note:} Unbiased equations of these two indicators are slightly more complex, but computer packages provide them automatically.

\section{Assumptions of Mean-Variance Analysis}
\begin{itemize}
    \item Allows to describe different assets simply.
    \item Assumes returns are normally distributed.
\end{itemize}
\paragraph{Normal distribution}
\begin{itemize}
    \item Its mean and median are equal;
    \item It's defined by two parameters, mean and variance;
    \item It's defined around its mean with:
    \begin{itemize}
        \item 68\% of observations within $\pm$ 1$\sigma$ of the mean.
        \item 95\% of observations within $\pm$ 2$\sigma$ of the mean.
        \item 99\% of observations within $\pm$ 3$\sigma$ of the mean
    \end{itemize}
    \item Returns are not normally distributed.
    \begin{itemize}
        \item Skewed: not symmetric around the mean.
        \item Characterized by high probability of extreme event.
    \end{itemize}
\end{itemize}

\section{Returns on Financial Assets}
\paragraph{Holding period return:} Return from holding an asset for a specific period of time.
\[R = \frac{P_t - P_{t-1}}{P_{t-1}} + \frac{D_t}{P_{t-1}}\]
Capital gain + Dividend yield \\
Holding period returns = compound returns
\[R = [(1 + r_1) \times (1 + r_2) \times (1+r_3)] - 1 \]

\paragraph{Geometric Mean Return}
\[\overline{R} = \sqrt[T]{(1+r_{i_1}) \times (1+r_{i_2}) \times ... \times (1+r_{i_T})} - 1\]
\paragraph{Annualized Return} These returns should be capitalized.
\begin{itemize}
    \item Such as:
    \[ r_{annual} = (1+r_{period})^C - 1\] with C being number of periods in a year.
\end{itemize}
\paragraph{Portfolio Return}
\begin{itemize}
    \item When several assets are combined into a portfolio, we can compute the portfolio return.
    \item Weighted average of the returns of individual assets
    \[ R_p = W_1 \times R_1 + W_2 \times R_2\]
\end{itemize}
\paragraph{Historical and Expected Returns}
\begin{itemize}
    \item Computed from historical data.
    \item What the investor expects to earn.
    \item Historical returns are different from expected returns.
\end{itemize}

\paragraph{Ways of Calculating Expected Returns}
\begin{itemize}
    \item Gut feeling;
    \item Modelling - calculated from a formula.
\end{itemize}

\paragraph{Standard deviation - Volatility}
\begin{itemize}
    \item Historical standard deviation;
    \item Defined as \textbf{risk} of equity return.
\end{itemize}

\paragraph*{Portfolio Variance}
\begin{itemize}
    \item Cannot simply add up two variances.
    \item Covariance:
    \[cov(X,Y) = E(XY) - E(X) \times E(Y)\]
    \begin{itemize}
        \item The more the two assets move in the same way, the higher the covariance.
        \item A negative covariance means the assets move in opposite directions.
    \end{itemize}
    \item Variance:
    \[ \sigma_v^2 = wCw^T \]
\end{itemize}

\paragraph*{Correlation and Portfolio Risk}
The correlation among assets determine the porfolio's risk.
\begin{itemize}
    \item Is a measure of tendency for N investments to act similarly;
    \item Can range from $-$1 and $+$1.
    \[\rho_{ij} = \frac{cov(R_i,R_j)}{\sigma_i \sigma_j}\]
\end{itemize}

\chapter{Lecture 2}

\end{document}